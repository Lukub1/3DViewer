\documentclass[a4paper,20pt]{article}

\usepackage{cmap}
\usepackage[utf8]{inputenc}
\usepackage[T2A]{fontenc}
\usepackage[english,russian]{babel}

\author{aelinorl, vileplum, maelyspe}
\title{3DViwer\_v1}
\date{}
\begin{document}
	\maketitle
	Данный проект реализован на языке программирования Си и Си++. Программа для просмотра 3D моделей в каркасном виде\footnote{Каркасная модель - модель объекта в трёхмерной графике, представляющая собой совокупность вершин и рёбер, которая определяет форму отображаемого многогранного объекта в трехмерном пространстве.} (3D Viewer). Сами модели необходимо загружать из файлов формата .obj\footnote{Файлы .obj - это формат файла описания геометрии, впервые разработанный компанией Wavefront Technologies. Формат файла открыт и принят многими поставщиками приложений для 3D-графики.}. Модели имеют возможность: вращения, масштабирования и перемещения.
	\newpage
	
	\subparagraph{Функции программы:}
	\flushleft
	\begin{itemize}
		\item Отображение каркасной 3D модели;
		\item Перемещение модели по 3-м осям (X,Y,Z);
		\item Поворот фигуры на заданный угол, от начального значения, по 3-м осям (X,Y,Z);
		\item Масштабирование модели.
	\end{itemize}
	
	\subparagraph{Пользовательский интерфейс:}
	\begin{itemize}
		\item Поле для отображения модели;
		\item Кнопка для выбора файла:
		\begin{itemize}
			\item[-] Открывается окно для выбора местоположения файла и самого файла.
		\end{itemize}
		\item Поле для отображения полного названия файла;
		\item Кнопка для отрисовки фигуры;
		\item Поле с отображением кол-ва вершин;
		\item Поле с отображением общего кол-ва ребер;
		\item Слайдер масштаба;
		\item 3 Слайдера для перемещения по осям X, Y и Z;
		\item 3 спинбокса для установки угла поворота по каждой оси;
		\item Кнопка для смены фона поля со всплывающим окном выбора цвета;
		\item бокс с параметрами линий состоящий из:
		\begin{itemize}
			\item[-] Кнопка для выбора цвета линии;
			\item[-] 2 кнопки выбора отрисовки линий;
			\item[-] Слайдер выбора толщины линии.
		\end{itemize}
		\item бокс с параметрами вершин состоящий из:
		\begin{itemize}
			\item[-] Кнопка для выбора цвета вершин;
			\item[-] 3 кнопки выбора отрисовки вершин;
			\item[-] Слайдер выбора размера вершин.
		\end{itemize}
		\item Кнопка "Сохранить" с выбором формата сохранения изображения.
		\item Кнопка для записи gif-анимации.
		
	\end{itemize}
	
\end{document}